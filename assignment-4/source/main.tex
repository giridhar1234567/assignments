\documentclass{article}
\usepackage{xcolor}
\usepackage{listings}

\definecolor{mGreen}{rgb}{0,0.6,0}
\definecolor{mGray}{rgb}{0.5,0.5,0.5}
\definecolor{mPurple}{rgb}{0.58,0,0.82}
\definecolor{backgroundColour}{rgb}{0.95,0.95,0.92}

\lstdefinestyle{CStyle}{
    backgroundcolor=\color{backgroundColour},   
    commentstyle=\color{mGreen},
    keywordstyle=\color{magenta},
    numberstyle=\tiny\color{mGray},
    stringstyle=\color{mPurple},
    basicstyle=\footnotesize,
    breakatwhitespace=false,         
    breaklines=true,                 
    captionpos=b,                    
    keepspaces=true,                 
    numbers=left,                    
    numbersep=5pt,                  
    showspaces=false,                
    showstringspaces=false,
    showtabs=false,                  
    tabsize=2,
    language=C
}



\begin{document}
\begin{lstlisting}[style=CStyle]
//Code written on December 5, 2020
//Revised  December 8, 2020
// by Giridhar paida
//This program implements a boolean function in C

#include <stdio.h>

//The  main function
int main(void)
{

//2 bits = 1 baud
//4 bits = 1 nibble
//8 bits = 1 byte

//unsigned char takes input as 1 byte

unsigned char  Z=0x00,Y=0x01,X=0x01,W=0x01;//inputs in hex	
unsigned char one = 0x01;//used for displaying the output in bit
unsigned char A,B,C,D;//outputs
A = ((~W)&(~X)&(~Y)&(~Z))|((~W)&(X)&(~Y)&(~Z))|((~W)&(~X)&Y&(~Z))|((~W)&X&Y&(~Z))|((~W)&(~X)&(~Y)&(Z));
//Boolean function for A
.
B = (W&(~X)&(~Y)&(~Z))|((~W)&X&(~Y)&(~Z))|(W&(~X)&Y&(~Z))|((~W)&X&Y&(~Z));//Boolean fuction

C = (W&X&(~Y)&(~Z))|((~W)&(~X)&Y&(~Z))|((W)&(~X)&Y&(~Z))|((~W)&X&Y&(~Z));//Boolean fuction

D = (W&X&Y&(~Z))|((~W)&(~X)&(~Y)&Z);//Boolean function for D
printf ("%x\n",one&A);// Output A 
printf ("%x\n",one&B);// OutputB
printf ("%x\n",one&C);// Output C
printf ("%x\n",one&D);// Output D

return 0;
}
[style=CStyle]
output
A = 0
B = 0
C = 0
D = 0
\end{lstlisting}
\end{document}
